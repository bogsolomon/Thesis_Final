\chapter{Conclusions} % Write in your own chapter title
\label{Chapter_conclusion}
\lhead{Chapter \ref{Chapter_conclusion}. \emph{Conclusions}} % Write in your own chapter title to set the page header

This thesis has presented a self-organizing, autonomic system responsible for managing cloud resources. The thesis has presented two algorithms for such a self-organizing system - one to detect the SLA breach and one to optimize the number of servers in the cloud. A test bed and a simulation were also created to run the application and validate the control system. Results were obtained, which prove that the control of the system can not be achieved by simply managing the CPU usage of the servers and that a more complex system, like the one presented in the thesis has to be used.

The novelty of the thesis comes from two separate parts which support each other:

\begin{enumerate}
	\item The ACO algorithm used in order to detect breaches of SLA. Normally the ACO is used in order to optimize the solution of a problem, however the thesis uses a modified version of the ACO in order to analyze the behaviour of the cloud, predict breaches of SLA and take corrective measures.
	\item A new self-organizing algorithm based on the behaviour of ants while house hunting is implemented and experiments to validate it are introduced. The algorithm is inspired by prior work and modeling into the behaviour of ants while house hunting, however there is no known algorithm which implements the behaviour in order to optimize a given problem.
\end{enumerate}

The thesis opens a number of avenues for future work and improvements to the new algorithms introduced:

\begin{enumerate}
	\item The house hunting algorithm can be improved to take in consideration the fitness of a solution when recruiting. While the fitness should not be the only factor in deciding which ant should recruit another, it can be used to bias positively solutions with better fitness. The current solution uses the fitness only to remove solutions with very poor fitness from the solution set.
	\item Further tests and simulations can be run on the created infrastructure in order to use more complex workloads and determine the behaviour of the proposed systems.
\end{enumerate}
